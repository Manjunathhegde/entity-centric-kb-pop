\section{Conclusion}
\label{sec:conclusion}

This paper presents \system{}, a simple but effective entity-centric framework for increasing knowledge densities in automatically constructed knowledge graphs. We find that \system{} is able to significantly increase NELL's knowledge density by a factor of 7.7 at 75.5\% accuracy. In addition to extracting new facts, \system{} is also able to extend the ontology. Our goal in this paper is twofold: (1) to draw attention to the effectiveness of entity-centric approaches with bigger scope (i.e., covering all four extraction classes in Table 1) towards improving knowledge density; and (2) to demonstrate that even relatively straightforward techniques can go a long way in alleviating low knowledge density in existing state-of- the-art KGs. 
Future work will include noise reduction in the pipeline. Close examining of sentences while constructing triples to make sure that noisy triples are not added to the system.

\system{} can be applied to other knowledge graphs with appropriate changes. Experiment with other normalization and relation mapping algorithms are the part of future work.
%we hope to experiment with more sophisticated algorithms as \system{}  components (e.g., normalization and linking modules) in hope of even better performance.

%This paper gives a simple but effective way of solving spareness of the knowledge bases.
%The quality of the extraction depends on the occurrence of the entity name in the web documents. Extractions for categories like Scientists, Universities and car manufacturers were better compared to books because
%of the same reason. 