\documentclass[conference]{IEEEtran}

\IEEEoverridecommandlockouts

\usepackage{setspace}
\usepackage{enumerate}
\usepackage{cite}
\usepackage{times}
\usepackage{url}
\usepackage{graphicx}
%\usepackage{subfigure}
\usepackage{amsmath}
\usepackage{algorithm}
\usepackage{algpseudocode}
\usepackage{pifont}
\usepackage{graphicx}
\usepackage{color}

\newcommand{\refalg}[1]{Algorithm~\ref{#1}}
\newcommand{\refeqn}[1]{Equation~\ref{#1}}
\newcommand{\reffig}[1]{Figure~\ref{#1}}
\newcommand{\reftbl}[1]{Table~\ref{#1}}
\newcommand{\refsec}[1]{Section~\ref{#1}}
\newcommand{\method}[1]{\mbox{\textsc{#1}}}

%\newcommand{\reminder}[1]{\textcolor{red}{[[  #1 ]]}\typeout{#1}}
\newcommand{\reminder}[1]{}

\newcommand{\system}{ENTICE}
\newcommand{\systemfull}{ENTIty Centric  Expansion}

\newtheorem{theorem}{Theorem}[section]
\newtheorem{claim}[theorem]{Claim}

\usepackage{array}
\newcolumntype{L}[1]{>{\raggedright\let\newline\\\arraybackslash\hspace{0pt}}m{#1}}
\newcolumntype{C}[1]{>{\centering\let\newline\\\arraybackslash\hspace{0pt}}m{#1}}
\newcolumntype{R}[1]{>{\raggedleft\let\newline\\\arraybackslash\hspace{0pt}}m{#1}}

\setlength\extrarowheight{2pt} 

\begin{document}
\title{An Entity-centric Approach for Overcoming \\Knowledge Graph Sparsity}
\author{
\IEEEauthorblockN{
Manjunath Hegde,
Partha Talukdar
}
\IEEEauthorblockA{Department~of~Computational~and~Data~Sciences\\
Indian Institute of Science, Bangalore, India\\
manjunath@ssl.serc.iisc.in,
ppt@cds.iisc.ac.in
}
}
\maketitle

\begin{abstract}

Automatic construction of knowledge graphs (KGs) from unstructured text has received considerable attention in recent research, resulting in the construction of several KGs with 
millions of entities (nodes) and facts (edges) among them. Unfortunately, such KGs tend to be severely sparse in terms of number of facts known for a \emph{given} entity, i.e., have low 
\emph{knowledge density}. For example, the NELL KG consists of only 1.34 facts per entity. Unfortunately, such low knowledge density makes it challenging to use  such KGs in real-world 
applications. In contrast to \emph{best-effort} extraction paradigms followed in the construction of such KGs, in this project we argue in favor of \systemfull{} (\system{}), an \emph{entity-centric}
KG population framework, to alleviate the low knowledge density problem in existing KGs. By using \system{}, we are able to increase NELL's knowledge density 
by a factor of 7.7 at 75.5\% accuracy. Additionally, we are also able to extend the ontology discovering new relations and entities. 
\end{abstract}

\section{Introduction}
\label{sec:intro}

%{\newcommand\T{\rule{0pt}{4ex}}


Over the last few years, automatic construction of knowledge graphs (KGs) from web-scale text data has received considerable attention, resulting in the construction of several large KGs such as NELL \cite{mitchell2015never}, Google's Knowledge Vault \cite{dong2014knowledge}. These KGs consist of millions of entities and facts involving them. While measuring size of the KGs in terms of number of entities and facts is helpful, they don't readily capture the volume of knowledge needed in real-world applications. When such a KG is used in an application, one is often interested in known facts for a \emph{given} entity, and not necessarily the overall size of the KG. In particular, knowing the average number of facts per entity is quite informative. We shall refer to this as the \textit{knowledge density} of the KG. %Please note that a sparse KG will also have low knowledge density.

Low knowledge density (or high sparsity) in automatically constructed KGs has been recognized in recent research \cite{west2014knowledge}. For example, NELL KG has a knowledge density of 1.34. Such low knowledge density puts significant limitations on the utility of these KGs. Construction of such KGs tend to follow a batch paradigm: the knowledge extraction system makes a full pass over the text corpus extracting whatever knowledge it finds, and finally aggregating all extractions into a graph. Clearly, such \emph{best-effort} extraction paradigm has proved to be inadequate to address the low knowledge density issue mentioned above. We refer to such paradigm as best-effort since its attention is divided equally among all possible entities.

\begin{table}[t]
	\begin{center}
		{\small
		\begin{tabular}{|p{1.5cm}|C{2.2cm}|C{1.8cm}|}
		\hline
		& {\bf K}nown Target {\bf E}ntity & {\bf N}ew Target {\bf E}ntity \\
		\hline
		%\noalign{\vskip 2mm}
		{\bf K}nown {\bf R}elation & KR-KE & KR-NE \\
		\hline
		{\bf N}ew {\bf R}elation & NR-KE & NR-NE \\
		\hline
		\end{tabular}
		}		
\caption{\label{tbl:extraction_taxonomy}Any  new fact involving a source entity from a Knowledge Graph (i.e., facts of the form \textit{entity1-relation-entity2} where  \textit{entity1} is already in the KG) can be classified into one of the  four extraction classes shown above. Most KG population techniques tend to focus on extracting facts of the KR-KE class. \system{}, the entity-centric approach proposed in this paper, is able to extract facts of all four classes.}
	\end{center}
\end{table}
%}

\begin{figure*}[!htbp]
\includegraphics[scale= 0.36]{images/pipeline.png}
\caption{Dataflow and architecture and  of \system{}. See \refsec{sec:method} for details.}
\label{fig:pipeline}
\end{figure*}

Recently, a few \emph{entity-centric} methods have been proposed to increase knowledge density in KGs \cite{gardner2013improving,gardner2014incorporating}. In contrast to the best-effort approaches mentioned above, these entity-centric approaches aim at increasing knowledge density for a \emph{given} entity. A new fact involving the  given entity can belong to one of the four types  shown in \reftbl{tbl:extraction_taxonomy}. Unfortunately, these densifying techniques only aim at identifying instances of known relations among entities already present in the KG, i.e., they fall in the KR-KE type of \reftbl{tbl:extraction_taxonomy}.

In this paper we propose \systemfull{} (\system{}), an entity-centric knowledge densifying framework which, given an entity, is capable of extracting facts belonging to all the four types shown in \reftbl{tbl:extraction_taxonomy}. By using \system{}, we are able to increase NELL's knowledge density by a factor of 7.7\footnote{Measured with respect to the five categories experimented with in the paper. See \refsec{sec:expts} for details.}, while achieving 75.4\% accuracy. %While the framework itself is relatively simple and can be improved,
Our goal here is to draw attention to the effectiveness of entity-centric approaches with bigger scope (i.e., covering all four extraction classes  in \reftbl{tbl:extraction_taxonomy}) towards improving knowledge density, and that even relatively straightforward techniques can go a long way in alleviating low knowledge density in existing state-of-the-art KGs. \system{} code is available at: {\small {\tt https://github.com/malllabiisc/entity-centric- kb-pop}} %\textit{https://github.com/Manjunathhegde/entity-centric-kb-pop}

% They either focus on identifying instances of of known relations among known entities in the KG \cite{gardner2013improving,gardner2014incorporating}, or they require proprietary technologies not available in the public domain \cite{west2014knowledge}.

%While the knowledge density may be computed directly from the number of entities and facts in the KG, we feel it is an important metric which needs to be reported independently. Knowledge densities of a few KGs are shown in 

%A large number of knowledge base (KB) construction projects have recently emerged. Prominent examples include Freebase which powers the Google Knowledge Graph, 
%ConceptNet, YAGO and others. These KBs contain many millions of entities, organized in hundreds to hundred thousands of semantic classes, and hundred millions of relational facts between entities. 
%The approach towards building these knowledge bases is to collect all possible information from the corpus and represent them in the KB. Our approach in constructing entity specific 
%knowledge base is to search for the predefined entities and extract facts related to that. We store the extractions in the form of triples. For the primary entity, extractions will be of the form - (primary entity, relation, entity-2)
%or (entity-2, relation, primary entity)
%Finding the (relation, entity-2) pairs for the given primary entity and classifying them as new or existing information is the aim of this paper. 
%The set of operations are carried out on the subject verb object triples to find the relations and entity-2.
%NELL has the facts related to lot of entities, hence we use it to classify extractions as new or existing fact.
%With the (relation, entity-2) pair gives 5 different combination.\\
%Case 1: both the relation and entity-2 exists in the knowledge base and they are connected. So it is a known fact.
%Case 2: both the relation and entity-2 exists in the knowledge base but they are not connected. So it is a new fact.
%Case 3: relation is present in the KB but entity-2 is new. So system has identified a new entity.
%Case 4: entity-2 is present in the KB but relation is new. So system has identified a new relation.
%Case 5: both entity-2 and relation are new.
%The other sections describe the related work on this topic, methodology in clustering the data and mapping to existing knowledge base.

\input{sections/related}


\begin{table*}[!htb]
\begin{small}
\begin{center}
\begin{tabular}{|p{2cm}|p{2cm}|p{2.5cm}|p{2cm}|p{2cm}|p{2cm}|p{2cm}|}
\hline
%\textbf{Category} & \textbf{Avg.No. of facts/entity in NELL} & \textbf{Avg.No. of facts/entity extracted} & \textbf{Facts Evaluated} & \textbf{No. of Correct Facts} & \textbf{Accuracy}\\
\textbf{Category} & \textbf{Knowledge Density in NELL} & \textbf{Knowledge Density after \system{}} & \textbf{\# Facts Evaluated} & \textbf{\# Correct Facts} & \textbf{Accuracy}\\
\hline
Scientist & 1.27 & 18.5 & 164 & 141 & 85.97 \\
\hline
Universities & 1.17 & 9 & 197 & 141 & 71.57\\
\hline
Books & 1.34 & 4.49 & 202 & 165 & 81.68 \\
\hline
Birds & 1.27 & 6.69 & 194 & 136 & 70.10 \\
\hline
Cars & 1.5 & 11.61 & 201 & 140 & 69.65 \\
\hline
\hline
Overall & 1.3 & 10.05 & 958 & 723 & 75.46 \\
\hline
\end{tabular}
\caption{\label{tbl:main_result}Knowledge densities of five categories in NELL and after application of \system{}, along with resulting accuracy. We observe that overall, \system{} is able to increase knowledge density by a factor of 7.7 at 75.5\% accuracy. This is our main result.}
\end{center}
%\end{table*}
%
%\begin{table*}[th]
%\begin{small}
\begin{tabular}{|p{2.3cm}|p{4.6cm}|p{5.5cm}|p{2cm}|}
%\begin{tabular}{|c|c|c|c|}
\hline
\textbf{Entity Name} & \textbf{All facts in NELL} & \textbf{Sample facts extracted by \system{}} & \textbf{Extraction Class} \\
\hline
%Seth Carlo Chandler & Seth Carlo Chandler can be found on Wikipedia at http://en.wikipedia.org/wiki/ Seth\%20Carlo\%20Chandler & Seth Carlo Chandler became associated with Benjamin Pierce & NR-KE\\ 
%\\
%{} & {} & Chandler used Almucantar & KR-KE \\
%\\
%{} & {} & Seth Carlo Chandler lived on Craigie street & KR-NE \\
%\hline
\textit{George Paget Thomson} & \textit{(George Paget Thomson, isInstanceOf, scientist)} & \textit{(Sir George Thomson, isFellowOf, Royal Society)} & NR-KE \\
%{} & {} & (George Thomson, married, Kathleen Buchanan Smith) & KR-NE \\
{} & {} & \textit{(George Thomson, hasSpouse, Kathleen Buchanan Smith)} & KR-NE \\
{} & {} & \textit{(George Paget Thomson, diedOn,  September 10)} & KR-KE\\
%\hline
%Joseph Engelberger & Joseph Engelberger is a scientist %Joseph Engelberger can be found on Wikipedia at http://en.wikipedia.org/wiki/ Joseph\%20Engelberger 
%	& Joseph Engelberger founded HelpMate Robotics Inc & KR-NE\\
%{} &  & Joseph Engelberger born in New York City & KR-KE\\
%{} & {} & Unimate actually predated Engelberger & NR-NE \\
\hline
\end{tabular}
\caption{\label{table:mapping}Facts corresponding to an entity from the \textit{scientists} domain in NELL as well as those extracted by \system{}. While NELL contained only one fact for this entity, \system{} was able to extract 15 facts for this entity, only 3 of which are shown above.}
%\vspace{0.2cm}
%\end{small}
%\end{table*}
%
%\begin{table*}[th]
%\begin{small}
\begin{tabular}{|p{1.5cm}|p{0.8cm}|p{0.8cm}|p{0.65cm}|p{0.8cm}|p{0.8cm}|p{0.65cm}|p{0.8cm}|p{0.8cm}|p{0.65cm}|p{0.8cm}|p{0.8cm}|p{0.65cm}|}
\hline
\textbf{Category} & \multicolumn{3}{c|}{\textbf{KR - KE}} & \multicolumn{3}{c|}{\textbf{KR - NE}}  & \multicolumn{3}{c|}{\textbf{NR - KE}} & \multicolumn{3}{c|}{\textbf{NR - NE}} \\
\hline
 {} & correct facts & wrong facts  & acc. & correct facts & wrong facts & acc. & correct facts & wrong facts & acc. & correct facts & wrong facts & acc.\\
\hline
Scientists & 57 & 10 & 85.07 & 61& 8 & 88.40 & 14 & 3 & 82.35 & 9 & 2 & 81.81 \\
\hline
Cars & 68 & 35 & 66.01 & 58 & 21 & 73.41 & 9 & 5 & 64.28 & 5 & 0 & 100 \\
\hline
Universities & 52 & 30 & 63.41 & 68 & 20 & 77.27 & 9 & 2 & 81.81 & 12 & 4 & 75 \\
\hline
Books & 78 & 24 & 76.47 & 79 & 12 & 86.81 & 2 & 0 & 100 & 6  & 1  & 85.71 \\
\hline
Birds & 67 & 29 & 69.79 & 46 & 19 & 70.76 & 15 & 4 & 78.94 & 8 & 6 & 57.14 \\
\hline
\hline
Overall & 322	 & 128 & 71.55 & 312 & 80 & 79.59 & 49 & 14 & 77.77 & 40 & 13 & 75.47 \\
\hline
\end{tabular}
\caption{\label{table:precision}Accuracy breakdown over \system{} extractions for each of the four extraction classes in \reftbl{tbl:extraction_taxonomy}. For each category, approximately 200 extractions were evaluated using Mechanical Turk.}
\end{small}
\end{table*}

\input{sections/method}
\section{Experiments}
\label{sec:expts}

In order to evaluate effectiveness of \system{}, we apply it to increase knowledge density for 100 randomly selected entities from each of the following five NELL categories: \textit{Scientist, Universities, Books, Birds}, and \textit{Cars}. For each category, a random subset of extractions in that category was evaluated using Mechanical Turk. To get a better accuracy of the evaluation, each fact was evaluated by 3 workers. Workers were made to classify each fact as correct, incorrect or can't say. 
Only those facts classified as correct by 2 or more evaluators were considered as correct facts.

{\bf Main Result}: Experimental results comparing knowledge densities in NELL and after application of \system{}, along with the accuracy of extractions, are presented in \reftbl{tbl:main_result}. 
From this, we observe that \system{} is able to improve knowledge density in NELL by a factor of 7.7 while maintaining 75.5\% accuracy.
Sample extraction examples and accuracy per-extraction class are presented in \reftbl{table:mapping} and \reftbl{table:precision}, respectively.

{\bf Noun and Relation Phrase Normalization}: We didn't perform any intrinsic evaluation of the entity and relation normalization step. However, in this section, we provide a few anecdotal examples to give a sense of the output quality from this step. We observe that the canopy clustering algorithm for entity and normalization is able to cluster together facts with somewhat different surface representations. For example, the algorithm came up with the following cluster with two facts: \textit{ \{(J. Willard Milnor, was awarded, 2011 Abel Prize); (John Milnor, received, Abel Prize)\}}. It is encouraging to see that the system is able to put \textit{J. Willard Milnor} and \textit{John Milnor} together, even though they have somewhat different surface forms (only one word overlap). Similarly, the relation phrases \textit{was awarded} and \textit{received} are also considered to be equivalent in the context of these beliefs.
%{\color{blue} It is interesting to see the type of clusters formed and reason for those cluster formation. The detailed evaluation of the clustering and normalization stage were not done, but the overall quality of the cluster formation and performance of the system was found to be satisfactory.
%For example in the below cluster, \\
%\textit{ \{(J. Willard Milnor, was awarded, 2011 Abel Prize); (John Milnor, received, Abel Prize)\}}\\
%The system has able to cluster facts which have different wordings in relation and entity phrases. In entity matching step, cases where one noun phrase is a substring of other, as in \textit{Abel Prize} and \textit{2011 Abel Prize} are put together.
%This works well in most of the cases but formation of clusters like \textit{1.\{(Indian), (Indian Institute of Science)\}} or \textit{2. \{(America and China),(America)\}} may not be a correct ones.
%
%Consider the example below, \\
%\textit{(Georg Waldemar Cantor, died on, January 6) \\
%(Georg Cantor,died on, January 1918)\\
%}
%These similar facts were placed in different clusters as the number part of the noun phrases are not matching. The canopy method with word level similarity matching may not be suitable for numbers and dates.
%} 

{\bf Integrating with Knowledge Graph}: Based on evaluation over a random-sampling, we find that entity linking in \system{} is 92\% accurate, while relation linking is about 70\% accurate.

In the entity linking stage, adjectives present in a noun phrase (NP) were ignored while matching the noun phrase to entities in the knowledge graph (NELL KB in this case). In case the whole NP didn't find any match, part of the NP was used to retrieve its category, if any. For example, in \textit{(Georg Waldemar Cantor, was born in, 1854)}, the NP  \textit{Georg Waldemar Cantor} was mapped to category \textit{person} using his last name and \textit{1854} to category \textit{date}. The relation phrase \textit{"was born in"} maps to many predicates in NELL relational metadata. NELL predicate \textit{AtDate} was selected based on the rule that category signature of the predicate matches the category of the noun phrases present in the triple. It also has the highest frequency count for the relational phrase in the metadata.

We observed that relation mapping has lesser accuracy due to two reasons. Firstly, error in determining right categories of NPs present in a triple; and secondly, due to higher ambiguity involving relation phrases in general, i.e., a single relation phrase usually matches many relation predicates in the ontology.

%{\color{blue}  In entity linking stage, titles and adjective of the noun phrases were removed while matching it to NELL. Part of the noun phrases were used to get the type in the cases were whole noun phase has no mention in NELL.
%In the following example, \\
%\textit{(Georg Waldemar Cantor, was born in (AtDate), 1854)\\}
%noun phrase \textit{Georg Waldemar Cantor} was mapped to category \textit{person} using his last name and \textit{1854} to category \textit{date}. The relation phrase \textit{was born in} maps to many predicates in NELL relational metadata.
%Predicate \textbf{AtDate} was selected based on the rule that type signature of the predicate matches the type of the noun phrases present in the fact. It also has the highest frequency count for the relational phrase in the metadata.
%In the example below,\\
%\textit{(Christiaan Eijkman, married (SpecializationOf), Aaltje Wigeri Edema)}\\
%\textit{married} has been mapped to \textit{SpecializationOf} because of error in finding the type for \textit{Aaltje Wigeri Edema}
%Relation mapping has lesser accuracy due to 2 reasons. 1. Error in getting type (category) of an entity and 2. Error in selecting a predicate when relation phrase matches many predicates in the metadata.
%}


%For our experiments, NELL knowledge base is chosen for comparison and mapping. Entities for the experiments are selected from NELL. 5 categories are chosen so as to test the working of 
%the system on different  kinds of entities. Subset of nearly 100 entities are randomly selected from NELL's collection. 
%The set of operations discussed in the previous section are performed on the selected entities. The number of queries made to get the entity related links was limited by query quota. Less than 20 documents per entity were selected for further processing. Few examples of the fact extracted are given in the table.  

%Only those extractions having the mention of primary entity are retained in the final result set. These extractions are evaluated through mechanical turk and majority of the answer is considered to assign label. These facts are again evaluated to ensure that labels are correct. 
%The  results from the evaluation and summary of all the facts for different categories are given in the table \ref{table:precision}.

%One more important outcome of these experiments was the category specific relations. After looking at all the extractions of a particular category, we were able to extract relational phrases which were
%highly related to category under observation. Only those phrases which occurs with high frequency were selected and tabulated below. Due to space constraints only few relations for only 2 categories are shown

%The mapping of the extraction to an existing knowledge base was a way to classify the extractions. The classification of the facts were done based on the novelty of the relation and entity as
%discussed in the introduction section. 
%Table \ref{table:mapping} gives few sample outputs of facts that were mapped to NELL. 


%The initial goal of this work was to reduce the sparsity of the knowledge bases by actively populating the facts at entity level. The extractions are classified based on their novelty and correctness of the 
%facts in each class is studied. It is also important to maintain the quality of the facts while expanding the knowledge graph. In our work we see that on an average the increase in the facts are x times 
%with a y \% precision. Table \ref{table:precision} summarizes precision and recall of the facts extracted.
%
%
%
%Table \ref{table:recall} summarizes the average number of facts per entity before and after running our extractor system.



\section{Conclusion}
\label{sec:conclusion}

This paper presents \system{}, a simple but effective entity-centric framework for increasing knowledge densities in automatically constructed knowledge graphs. We find that \system{} is able to significantly increase NELL's knowledge density by a factor of 7.7 at 75.5\% accuracy. In addition to extracting new facts, \system{} is also able to extend the ontology. Our goal in this paper is twofold: (1) to draw attention to the effectiveness of entity-centric approaches with bigger scope (i.e., covering all four extraction classes in Table 1) towards improving knowledge density; and (2) to demonstrate that even relatively straightforward techniques can go a long way in alleviating low knowledge density in existing state-of- the-art KGs. 
Future work will include noise reduction in the pipeline. Close examining of sentences while constructing triples to make sure that noisy triples are not added to the system.

\system{} can be applied to other knowledge graphs with appropriate changes. Experiment with other normalization and relation mapping algorithms are the part of future work.
%we hope to experiment with more sophisticated algorithms as \system{}  components (e.g., normalization and linking modules) in hope of even better performance.

%This paper gives a simple but effective way of solving spareness of the knowledge bases.
%The quality of the extraction depends on the occurrence of the entity name in the web documents. Extractions for categories like Scientists, Universities and car manufacturers were better compared to books because
%of the same reason. 
\section*{Acknowledgment}
I would like to thank Dr. Partha Talukdar for his valuable guidance and support throught the project.
Special thanks to Tushar Nagarajan for building the CNN model and helping me throught the relation mapping process.
All the other members of the Machine and Language Learning (MALL) lab friends for their suggestions and timely help.
I thank MALL lab and Computational and Data Science Department for providing Computational facilities.

\bibliographystyle{IEEEtran}
\bibliography{template}

\end{document}

